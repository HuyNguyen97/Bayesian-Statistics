\documentclass[12pt]{article} 
% Custom definitions
% To use this customization file, insert the line "% Custom definitions
% To use this customization file, insert the line "% Custom definitions
% To use this customization file, insert the line "\input{custom}" in the header of the tex file.

% Formatting

\tolerance=1000
\usepackage[margin=1in]{geometry}


% Packages

% \usepackage{amssymb,latexsym}
\usepackage{amssymb,amsfonts,amsmath,latexsym,amsthm}
\usepackage[usenames,dvipsnames]{color}
\usepackage[]{graphicx}
\usepackage[space]{grffile}
\usepackage{mathrsfs}   % fancy math font
% \usepackage[font=small,skip=0pt]{caption}
\usepackage[skip=0pt]{caption}
\usepackage{subcaption}
\usepackage{verbatim}
\usepackage{url}
\usepackage{bm}
\usepackage{dsfont}
\usepackage{extarrows}
\usepackage{multirow}
% \usepackage{wrapfig}
% \usepackage{epstopdf}
\usepackage{rotating}
\usepackage{tikz}
\usetikzlibrary{fit}					% fitting shapes to coordinates
%\usetikzlibrary{backgrounds}	% drawing the background after the foreground

\usepackage{fancyhdr}

\fancypagestyle{firststyle}{
   \fancyhf{}
   \renewcommand{\footrulewidth}{0.4pt}
   \fancyfoot[L]{\footnotesize This work is licensed under a \href{http://creativecommons.org/licenses/by-nc-nd/4.0/}{Creative Commons BY-NC-ND 4.0 International License}.\\ Jeffrey W. Miller (2015). \textit{Lecture Notes on Bayesian Statistics}. Duke University, Durham, NC.}
}


% \usepackage[dvipdfm,colorlinks,citecolor=blue,linkcolor=blue,urlcolor=blue]{hyperref}
\usepackage[colorlinks,citecolor=blue,linkcolor=blue,urlcolor=blue]{hyperref}
%\usepackage{hyperref}
\usepackage[authoryear,round]{natbib}


%  Theorems, etc.

\theoremstyle{plain}
\newtheorem{theorem}{Theorem}[section]
\newtheorem{corollary}[theorem]{Corollary}
\newtheorem{lemma}[theorem]{Lemma}
\newtheorem{proposition}[theorem]{Proposition}
\newtheorem{condition}[theorem]{Condition}
% \newtheorem{conditions}[theorem]{Conditions}

\theoremstyle{definition}
\newtheorem{definition}[theorem]{Definition}
% \newtheorem*{unnumbered-definition}{Definition}
\newtheorem{example}[theorem]{Example}
\theoremstyle{remark}
\newtheorem*{remark}{Remark}
\numberwithin{equation}{section}



% Document-specific shortcuts
\newcommand{\btheta}{{\bm\theta}}
\newcommand{\bbtheta}{{\pmb{\bm\theta}}}

\newcommand{\commentary}[1]{\ifx\showcommentary\undefined\else \emph{#1}\fi}

\newcommand{\term}[1]{\textit{\textbf{#1}}}

% Math shortcuts

% Probability distributions
\DeclareMathOperator*{\Exp}{Exp}
\DeclareMathOperator*{\TExp}{TExp}
\DeclareMathOperator*{\Bernoulli}{Bernoulli}
\DeclareMathOperator*{\Beta}{Beta}
\DeclareMathOperator*{\Ga}{Gamma}
\DeclareMathOperator*{\TGamma}{TGamma}
\DeclareMathOperator*{\Poisson}{Poisson}
\DeclareMathOperator*{\Binomial}{Binomial}
\DeclareMathOperator*{\NormalGamma}{NormalGamma}
\DeclareMathOperator*{\InvGamma}{InvGamma}
\DeclareMathOperator*{\Cauchy}{Cauchy}
\DeclareMathOperator*{\Uniform}{Uniform}
\DeclareMathOperator*{\Gumbel}{Gumbel}
\DeclareMathOperator*{\Pareto}{Pareto}
\DeclareMathOperator*{\Mono}{Mono}
\DeclareMathOperator*{\Geometric}{Geometric}
\DeclareMathOperator*{\Wishart}{Wishart}

% Math operators
\DeclareMathOperator*{\argmin}{arg\,min}
\DeclareMathOperator*{\argmax}{arg\,max}
\DeclareMathOperator*{\Cov}{Cov}
\DeclareMathOperator*{\diag}{diag}
\DeclareMathOperator*{\median}{median}
\DeclareMathOperator*{\Vol}{Vol}

% Math characters
\newcommand{\R}{\mathbb{R}}
\newcommand{\Z}{\mathbb{Z}}
\newcommand{\E}{\mathbb{E}}
\renewcommand{\Pr}{\mathbb{P}}
\newcommand{\I}{\mathds{1}}
\newcommand{\V}{\mathbb{V}}

\newcommand{\A}{\mathcal{A}}
\newcommand{\C}{\mathcal{C}}
\newcommand{\D}{\mathcal{D}}
\newcommand{\Hcal}{\mathcal{H}}
\newcommand{\M}{\mathcal{M}}
\newcommand{\N}{\mathcal{N}}
\newcommand{\X}{\mathcal{X}}
\newcommand{\Zcal}{\mathcal{Z}}
\renewcommand{\P}{\mathcal{P}}

\newcommand{\T}{\mathtt{T}}
\renewcommand{\emptyset}{\varnothing}


% Miscellaneous commands
\newcommand{\iid}{\stackrel{\mathrm{iid}}{\sim}}
\newcommand{\matrixsmall}[1]{\bigl(\begin{smallmatrix}#1\end{smallmatrix} \bigr)}

\newcommand{\items}[1]{\begin{itemize} #1 \end{itemize}}

\newcommand{\todo}[1]{\emph{\textcolor{red}{(#1)}}}

\newcommand{\branch}[4]{
\left\{
	\begin{array}{ll}
		#1  & \mbox{if } #2 \\
		#3 & \mbox{if } #4
	\end{array}
\right.
}

% approximately proportional to
\def\app#1#2{%
  \mathrel{%
    \setbox0=\hbox{$#1\sim$}%
    \setbox2=\hbox{%
      \rlap{\hbox{$#1\propto$}}%
      \lower1.3\ht0\box0%
    }%
    \raise0.25\ht2\box2%
  }%
}
\def\approxprop{\mathpalette\app\relax}

% \newcommand{\approptoinn}[2]{\mathrel{\vcenter{
  % \offinterlineskip\halign{\hfil$##$\cr
    % #1\propto\cr\noalign{\kern2pt}#1\sim\cr\noalign{\kern-2pt}}}}}

% \newcommand{\approxpropto}{\mathpalette\approptoinn\relax}





" in the header of the tex file.

% Formatting

\tolerance=1000
\usepackage[margin=1in]{geometry}


% Packages

% \usepackage{amssymb,latexsym}
\usepackage{amssymb,amsfonts,amsmath,latexsym,amsthm}
\usepackage[usenames,dvipsnames]{color}
\usepackage[]{graphicx}
\usepackage[space]{grffile}
\usepackage{mathrsfs}   % fancy math font
% \usepackage[font=small,skip=0pt]{caption}
\usepackage[skip=0pt]{caption}
\usepackage{subcaption}
\usepackage{verbatim}
\usepackage{url}
\usepackage{bm}
\usepackage{dsfont}
\usepackage{extarrows}
\usepackage{multirow}
% \usepackage{wrapfig}
% \usepackage{epstopdf}
\usepackage{rotating}
\usepackage{tikz}
\usetikzlibrary{fit}					% fitting shapes to coordinates
%\usetikzlibrary{backgrounds}	% drawing the background after the foreground

\usepackage{fancyhdr}

\fancypagestyle{firststyle}{
   \fancyhf{}
   \renewcommand{\footrulewidth}{0.4pt}
   \fancyfoot[L]{\footnotesize This work is licensed under a \href{http://creativecommons.org/licenses/by-nc-nd/4.0/}{Creative Commons BY-NC-ND 4.0 International License}.\\ Jeffrey W. Miller (2015). \textit{Lecture Notes on Bayesian Statistics}. Duke University, Durham, NC.}
}


% \usepackage[dvipdfm,colorlinks,citecolor=blue,linkcolor=blue,urlcolor=blue]{hyperref}
\usepackage[colorlinks,citecolor=blue,linkcolor=blue,urlcolor=blue]{hyperref}
%\usepackage{hyperref}
\usepackage[authoryear,round]{natbib}


%  Theorems, etc.

\theoremstyle{plain}
\newtheorem{theorem}{Theorem}[section]
\newtheorem{corollary}[theorem]{Corollary}
\newtheorem{lemma}[theorem]{Lemma}
\newtheorem{proposition}[theorem]{Proposition}
\newtheorem{condition}[theorem]{Condition}
% \newtheorem{conditions}[theorem]{Conditions}

\theoremstyle{definition}
\newtheorem{definition}[theorem]{Definition}
% \newtheorem*{unnumbered-definition}{Definition}
\newtheorem{example}[theorem]{Example}
\theoremstyle{remark}
\newtheorem*{remark}{Remark}
\numberwithin{equation}{section}



% Document-specific shortcuts
\newcommand{\btheta}{{\bm\theta}}
\newcommand{\bbtheta}{{\pmb{\bm\theta}}}

\newcommand{\commentary}[1]{\ifx\showcommentary\undefined\else \emph{#1}\fi}

\newcommand{\term}[1]{\textit{\textbf{#1}}}

% Math shortcuts

% Probability distributions
\DeclareMathOperator*{\Exp}{Exp}
\DeclareMathOperator*{\TExp}{TExp}
\DeclareMathOperator*{\Bernoulli}{Bernoulli}
\DeclareMathOperator*{\Beta}{Beta}
\DeclareMathOperator*{\Ga}{Gamma}
\DeclareMathOperator*{\TGamma}{TGamma}
\DeclareMathOperator*{\Poisson}{Poisson}
\DeclareMathOperator*{\Binomial}{Binomial}
\DeclareMathOperator*{\NormalGamma}{NormalGamma}
\DeclareMathOperator*{\InvGamma}{InvGamma}
\DeclareMathOperator*{\Cauchy}{Cauchy}
\DeclareMathOperator*{\Uniform}{Uniform}
\DeclareMathOperator*{\Gumbel}{Gumbel}
\DeclareMathOperator*{\Pareto}{Pareto}
\DeclareMathOperator*{\Mono}{Mono}
\DeclareMathOperator*{\Geometric}{Geometric}
\DeclareMathOperator*{\Wishart}{Wishart}

% Math operators
\DeclareMathOperator*{\argmin}{arg\,min}
\DeclareMathOperator*{\argmax}{arg\,max}
\DeclareMathOperator*{\Cov}{Cov}
\DeclareMathOperator*{\diag}{diag}
\DeclareMathOperator*{\median}{median}
\DeclareMathOperator*{\Vol}{Vol}

% Math characters
\newcommand{\R}{\mathbb{R}}
\newcommand{\Z}{\mathbb{Z}}
\newcommand{\E}{\mathbb{E}}
\renewcommand{\Pr}{\mathbb{P}}
\newcommand{\I}{\mathds{1}}
\newcommand{\V}{\mathbb{V}}

\newcommand{\A}{\mathcal{A}}
\newcommand{\C}{\mathcal{C}}
\newcommand{\D}{\mathcal{D}}
\newcommand{\Hcal}{\mathcal{H}}
\newcommand{\M}{\mathcal{M}}
\newcommand{\N}{\mathcal{N}}
\newcommand{\X}{\mathcal{X}}
\newcommand{\Zcal}{\mathcal{Z}}
\renewcommand{\P}{\mathcal{P}}

\newcommand{\T}{\mathtt{T}}
\renewcommand{\emptyset}{\varnothing}


% Miscellaneous commands
\newcommand{\iid}{\stackrel{\mathrm{iid}}{\sim}}
\newcommand{\matrixsmall}[1]{\bigl(\begin{smallmatrix}#1\end{smallmatrix} \bigr)}

\newcommand{\items}[1]{\begin{itemize} #1 \end{itemize}}

\newcommand{\todo}[1]{\emph{\textcolor{red}{(#1)}}}

\newcommand{\branch}[4]{
\left\{
	\begin{array}{ll}
		#1  & \mbox{if } #2 \\
		#3 & \mbox{if } #4
	\end{array}
\right.
}

% approximately proportional to
\def\app#1#2{%
  \mathrel{%
    \setbox0=\hbox{$#1\sim$}%
    \setbox2=\hbox{%
      \rlap{\hbox{$#1\propto$}}%
      \lower1.3\ht0\box0%
    }%
    \raise0.25\ht2\box2%
  }%
}
\def\approxprop{\mathpalette\app\relax}

% \newcommand{\approptoinn}[2]{\mathrel{\vcenter{
  % \offinterlineskip\halign{\hfil$##$\cr
    % #1\propto\cr\noalign{\kern2pt}#1\sim\cr\noalign{\kern-2pt}}}}}

% \newcommand{\approxpropto}{\mathpalette\approptoinn\relax}





" in the header of the tex file.

% Formatting

\tolerance=1000
\usepackage[margin=1in]{geometry}


% Packages

% \usepackage{amssymb,latexsym}
\usepackage{amssymb,amsfonts,amsmath,latexsym,amsthm}
\usepackage[usenames,dvipsnames]{color}
\usepackage[]{graphicx}
\usepackage[space]{grffile}
\usepackage{mathrsfs}   % fancy math font
% \usepackage[font=small,skip=0pt]{caption}
\usepackage[skip=0pt]{caption}
\usepackage{subcaption}
\usepackage{verbatim}
\usepackage{url}
\usepackage{bm}
\usepackage{dsfont}
\usepackage{extarrows}
\usepackage{multirow}
% \usepackage{wrapfig}
% \usepackage{epstopdf}
\usepackage{rotating}
\usepackage{tikz}
\usetikzlibrary{fit}					% fitting shapes to coordinates
%\usetikzlibrary{backgrounds}	% drawing the background after the foreground

\usepackage{fancyhdr}

\fancypagestyle{firststyle}{
   \fancyhf{}
   \renewcommand{\footrulewidth}{0.4pt}
   \fancyfoot[L]{\footnotesize This work is licensed under a \href{http://creativecommons.org/licenses/by-nc-nd/4.0/}{Creative Commons BY-NC-ND 4.0 International License}.\\ Jeffrey W. Miller (2015). \textit{Lecture Notes on Bayesian Statistics}. Duke University, Durham, NC.}
}


% \usepackage[dvipdfm,colorlinks,citecolor=blue,linkcolor=blue,urlcolor=blue]{hyperref}
\usepackage[colorlinks,citecolor=blue,linkcolor=blue,urlcolor=blue]{hyperref}
%\usepackage{hyperref}
\usepackage[authoryear,round]{natbib}


%  Theorems, etc.

\theoremstyle{plain}
\newtheorem{theorem}{Theorem}[section]
\newtheorem{corollary}[theorem]{Corollary}
\newtheorem{lemma}[theorem]{Lemma}
\newtheorem{proposition}[theorem]{Proposition}
\newtheorem{condition}[theorem]{Condition}
% \newtheorem{conditions}[theorem]{Conditions}

\theoremstyle{definition}
\newtheorem{definition}[theorem]{Definition}
% \newtheorem*{unnumbered-definition}{Definition}
\newtheorem{example}[theorem]{Example}
\theoremstyle{remark}
\newtheorem*{remark}{Remark}
\numberwithin{equation}{section}



% Document-specific shortcuts
\newcommand{\btheta}{{\bm\theta}}
\newcommand{\bbtheta}{{\pmb{\bm\theta}}}

\newcommand{\commentary}[1]{\ifx\showcommentary\undefined\else \emph{#1}\fi}

\newcommand{\term}[1]{\textit{\textbf{#1}}}

% Math shortcuts

% Probability distributions
\DeclareMathOperator*{\Exp}{Exp}
\DeclareMathOperator*{\TExp}{TExp}
\DeclareMathOperator*{\Bernoulli}{Bernoulli}
\DeclareMathOperator*{\Beta}{Beta}
\DeclareMathOperator*{\Ga}{Gamma}
\DeclareMathOperator*{\TGamma}{TGamma}
\DeclareMathOperator*{\Poisson}{Poisson}
\DeclareMathOperator*{\Binomial}{Binomial}
\DeclareMathOperator*{\NormalGamma}{NormalGamma}
\DeclareMathOperator*{\InvGamma}{InvGamma}
\DeclareMathOperator*{\Cauchy}{Cauchy}
\DeclareMathOperator*{\Uniform}{Uniform}
\DeclareMathOperator*{\Gumbel}{Gumbel}
\DeclareMathOperator*{\Pareto}{Pareto}
\DeclareMathOperator*{\Mono}{Mono}
\DeclareMathOperator*{\Geometric}{Geometric}
\DeclareMathOperator*{\Wishart}{Wishart}

% Math operators
\DeclareMathOperator*{\argmin}{arg\,min}
\DeclareMathOperator*{\argmax}{arg\,max}
\DeclareMathOperator*{\Cov}{Cov}
\DeclareMathOperator*{\diag}{diag}
\DeclareMathOperator*{\median}{median}
\DeclareMathOperator*{\Vol}{Vol}

% Math characters
\newcommand{\R}{\mathbb{R}}
\newcommand{\Z}{\mathbb{Z}}
\newcommand{\E}{\mathbb{E}}
\renewcommand{\Pr}{\mathbb{P}}
\newcommand{\I}{\mathds{1}}
\newcommand{\V}{\mathbb{V}}

\newcommand{\A}{\mathcal{A}}
\newcommand{\C}{\mathcal{C}}
\newcommand{\D}{\mathcal{D}}
\newcommand{\Hcal}{\mathcal{H}}
\newcommand{\M}{\mathcal{M}}
\newcommand{\N}{\mathcal{N}}
\newcommand{\X}{\mathcal{X}}
\newcommand{\Zcal}{\mathcal{Z}}
\renewcommand{\P}{\mathcal{P}}

\newcommand{\T}{\mathtt{T}}
\renewcommand{\emptyset}{\varnothing}


% Miscellaneous commands
\newcommand{\iid}{\stackrel{\mathrm{iid}}{\sim}}
\newcommand{\matrixsmall}[1]{\bigl(\begin{smallmatrix}#1\end{smallmatrix} \bigr)}

\newcommand{\items}[1]{\begin{itemize} #1 \end{itemize}}

\newcommand{\todo}[1]{\emph{\textcolor{red}{(#1)}}}

\newcommand{\branch}[4]{
\left\{
	\begin{array}{ll}
		#1  & \mbox{if } #2 \\
		#3 & \mbox{if } #4
	\end{array}
\right.
}

% approximately proportional to
\def\app#1#2{%
  \mathrel{%
    \setbox0=\hbox{$#1\sim$}%
    \setbox2=\hbox{%
      \rlap{\hbox{$#1\propto$}}%
      \lower1.3\ht0\box0%
    }%
    \raise0.25\ht2\box2%
  }%
}
\def\approxprop{\mathpalette\app\relax}

% \newcommand{\approptoinn}[2]{\mathrel{\vcenter{
  % \offinterlineskip\halign{\hfil$##$\cr
    % #1\propto\cr\noalign{\kern2pt}#1\sim\cr\noalign{\kern-2pt}}}}}

% \newcommand{\approxpropto}{\mathpalette\approptoinn\relax}








\begin{document}
\begin{center}
\large\textbf{STA360/601 Midterm Exam}
\end{center}

\small

\subsection*{Instructions}
\begin{itemize}
    \item Write your name, NetID, and signature below.
    \item If you need extra space for any problem, continue on the back of the page.
\end{itemize}

\subsection*{Community Standard}
To uphold the Duke Community Standard:
\begin{itemize}
\item I will not lie, cheat, or steal in my academic endeavors;
\item I will conduct myself honorably in all my endeavors; and
\item I will act if the Standard is compromised.
\end{itemize}
I have adhered to the Duke Community Standard in completing this exam.

\vspace{1em}
\begin{itemize}
    \setlength\itemsep{1em}
    \item[] Name: \hrulefill
    \item[] NetID: \hrulefill
    \item[] Signature: \hrulefill
\end{itemize}

\subsection*{Score}
(For TA use only --- leave this section blank.)

\vspace{1em}
%\begin{minipage}{1.0\textwidth}
\begin{enumerate}
    \setlength\itemsep{1em}
    \item \line(1,0){100}
    \item \line(1,0){100}
    \item \line(1,0){100}
    \item \line(1,0){100}
    \item \line(1,0){100}
    \item \line(1,0){100}
    %\item \line(1,0){100}
        \vspace{1em}
    \item[] Overall: \line(1,0){200}
\end{enumerate}

\newpage
\subsection*{List of common distributions}
% todo: any others?
\begin{itemize}
    \setlength\itemsep{1em}
    \item[] $\displaystyle \Geometric(x|\theta) = \theta(1-\theta)^x\,\I(x\in\{0,1,2,\ldots\})$ for $0<\theta<1$
    \item[] $\displaystyle \Bernoulli(x|\theta) = \theta^x(1-\theta)^{1-x}\,\I(x\in\{0,1\})$ for $0<\theta<1$
    \item[] $\displaystyle \Binomial(x|n,\theta) = {n\choose x}\theta^x(1-\theta)^{n-x}\,\I(x\in\{0,1,\ldots,n\})$ for $0<\theta<1$
    \item[] $\displaystyle \Poisson(x|\theta) = \frac{e^{-\theta}\theta^x}{x!}\,\I(x\in\{0,1,2,\ldots\})$ for $\theta>0$
    \item[] $\displaystyle \Exp(x|\theta) = \theta e^{-\theta x}\,\I(x>0)$ for $\theta>0$
    \item[] $\displaystyle \Uniform(x|a,b) = \frac{1}{b-a}\,\I(a<x<b)$ for $a<b$
    \item[] $\displaystyle \Ga(x|a,b) = \frac{b^a}{\Gamma(a)}x^{a-1}e^{-b x}\,\I(x>0)$ for $a,b>0$
    \item[] $\displaystyle \Pareto(x|\alpha,c) = \frac{\alpha c^\alpha}{x^{\alpha+1}}\,\I(x>c)$ for $\alpha,c>0$
    \item[] $\displaystyle \Beta(x|a,b) = \frac{1}{B(a,b)}x^{a-1}(1-x)^{b-1}\,\I(0<x<1)$ for $a,b>0$
    \item[] $\displaystyle \N(x|\mu,\sigma^2) = \frac{1}{\sqrt{2\pi\sigma^2}}\exp\big(-\tfrac{1}{2\sigma^2}(x-\mu)^2\big)$
        for $\mu\in\R$, $\sigma^2>0$
    \item[] $\displaystyle \N(x|\mu,\lambda^{-1}) = \sqrt{\frac{\lambda}{2\pi}}\exp\big(-\tfrac{1}{2}\lambda(x-\mu)^2\big)$
        for $\mu\in\R$, $\lambda>0$
\end{itemize}

\subsection*{Exponential family form}
$$ p(x|\theta) =\exp\big(\varphi(\theta)^\T t(x)-\kappa(\theta)\big) h(x) $$

\subsection*{List of special functions}
\label{special-functions}
\begin{itemize}
    \setlength\itemsep{1em}
    \item[] Beta function: $\displaystyle B(a,b) = \int_0^1 t^{a-1}(1-t)^{b-1} d t$ for $a,b>0$
    \item[] Gamma function: $\displaystyle \Gamma(x) = \int_0^\infty t^{x-1} e^{-t} d t$ for $x>0$
    \item[] Log function for $a>1$: $\displaystyle \log a = \int_1^a (1/t) d t$
\end{itemize}


\normalsize

\newpage
\begin{enumerate}
\item (15 points)
    \begin{enumerate}
        \setlength\itemsep{4em}
        \item (5 points) What is the formula for Bayes' theorem?
        \item (5 points) You receive $x_i$ telephone calls on day $i$, for $i=1,\ldots,n$. 
            You wish to model this as $X_1,\ldots,X_n$ i.i.d.\ from some distribution.
            Which of the following distributions would make sense to use? (Circle one.)
            \vspace{1em}
            \begin{enumerate}
                \setlength\itemsep{1em}
                \item Beta
                \item Poisson
                \item Bernoulli
                \item Exponential
            \end{enumerate}
        \item (5 points) Suppose $X,X_1,\ldots,X_N$ are i.i.d.\ and assume $\E|X|<\infty$ and $\V(X)<\infty$.
            What is the standard deviation of $$\frac{1}{N}\sum_{i=1}^N X_i?$$
            Hint: It is the same as the RMSE of the Monte Carlo approximation. (Circle one.)
            \vspace{1em}
            \begin{enumerate}
                \setlength\itemsep{1em}
                \item $\V(X)/N$
                \item $\V(X)/\sqrt N$
                \item $\sigma(X)/N$
                \item $\sigma(X)/\sqrt N$
            \end{enumerate}
    \end{enumerate}

\newpage
\item (17 points) (Marginal likelihood)\\
    Suppose $X_1,\ldots,X_n\iid\Geometric(\theta)$ given $\theta$. Consider a $\Beta(a,b)$ prior on $\theta$. What is the
    marginal likelihood $p(x_{1:n})$?
    
    \vspace{1em}
    (Your answer must be an explicit expression in terms of $a,b,x_1,\ldots,x_n,n$, and any of the special
    functions on page \pageref{special-functions}. You must show your work to receive full credit.)

\newpage
\item (17 points) (Exponential families, Normal distribution)\\
    Show that the collection of $\N(\mu,\sigma^2)$ distributions, with $\mu\in\R$ and $\sigma^2>0$, is a
    two-parameter exponential family, and identify the sufficient statistics function $t(x)=(t_1(x),t_2(x))^\T$ for your parametrization.

\newpage
\item (17 points) (Conjugate priors)\\
    Suppose $X_1,\ldots,X_n\iid\Uniform(0,\theta)$ given $\theta$, that is,
    $$ p(x_i|\theta) = \frac{1}{\theta}\,\I(0<x_i<\theta). $$
    You would like to find a conjugate prior for $\theta$.
    Show that the family of $\Pareto(\alpha,c)$ distributions, with $\alpha>0$ and $c>0$, is a conjugate prior family.
    
\newpage
\item (17 points) (Sampling methods)\\
    Suppose $c>1$ and
    $$ p(x) \propto \frac{1}{x}\,\I(1<x<c). $$
    (Note that $p(x)$ is proportional to this, not equal to this.)
    Assume you can generate $U\sim\Uniform(0,1)$.  Give an explicit formula, in terms of $c$ and $U$, for generating a sample from $p(x)$.  You must show your work to receive full credit.

\newpage
\item (17 points) (Decision theory)\\
    Consider a decision problem in which the state is $\theta\in\R$, the observation is $x$, you must choose an action
    $\hat\theta\in\R$, and the loss function is
    $$\ell(\theta,\hat\theta) = a\theta^2 + b\theta\hat\theta + c\hat\theta^2$$
    for some known $a,b,c\in\R$ with $c>0$.
    Suppose you have computed the posterior distribution and it is $p(\theta|x) = \N(\theta|M,L^{-1})$ for some $M$ and $L$.
    What is the Bayes procedure (minimizing posterior expected loss)?

    \vspace{1em}
    (Your answer must be an explicit expression in terms of $a,b,c,M$, and $L$. You must show your work to receive full credit.)



% \newpage
% \item (14 points) (Monte Carlo and Importance sampling)\\
% Suppose $h(x) = x^2$.  Find the limit of each of the following expressions, as $N \to \infty$.
% Each answer must be a specific numerical value (not just a formula), and you must show your work to receive full credit.
    % \begin{enumerate}
        % \setlength\itemsep{11em}
        % \item (5 points) $\displaystyle\frac{1}{N}\sum_{i=1}^N h(X_i)$ where $X_1,X_2,\ldots\iid\Uniform(0,1)$.
        % \item (5 points) $\displaystyle\frac{1}{N}\sum_{i=1}^N h(\Phi(Z_i))$ where $Z_1,Z_2,\ldots\iid\N(0,1)$
            % and $\Phi(z)$ is the $\N(0,1)$ c.d.f.
        % \item (4 points) $\displaystyle\frac{1}{N}\sum_{i=1}^N \frac{h(Z_i)}{\varphi(Z_i)}\I(0<Z_i<1)$ where $Z_1,Z_2,\ldots\iid\N(0,1)$
            % and $\varphi(z)$ is the $\N(0,1)$ p.d.f.
    % \end{enumerate}

    
\end{enumerate}


\end{document}






